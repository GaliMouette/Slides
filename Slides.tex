\documentclass[french,usepdftitle=false,compress]{beamer}

\usepackage[french]{babel}
\usepackage[utf8]{inputenc}
\usepackage[T1]{fontenc}

\mode<presentation>
{
  \usetheme{Dresden}
  \usecolortheme{crane}
  \setbeamercovered{transparent}
  \setbeamertemplate{navigation symbols}{}
  \setbeamertemplate{footline}[frame number]
}

\title
{
  Application spontanée de transformations logiques en assistance au raisonnement automatique
}

\subtitle
{
  Stage de L3
}

\author[Alexis~CARRÉ]
{
  Alexis~CARRÉ \inst{1}\\[1em]
  {
    \tiny
    Sous la direction de :\\[-0.5em]
    Chantal~KELLER \inst{2} \hspace{1em} Louise~DUBOIS~DE~PRISQUE \inst{2}
  }
}

\institute[École Normale Supérieure de Lyon et Université Paris Saclay]
{
  \inst{1}
  École Normale Supérieure de Lyon
  \and
  \inst{2}
  Laboratoire Méthodes Formelles\\
  Université Paris Saclay
}

\date
{
  4 septembre 2023 : Soutenance
}

\begin{document}

\begin{frame}[noframenumbering, plain]
  \titlepage
\end{frame}

\section{Introduction}

\subsection{Les assistants de preuve}

\begin{frame}{La preuve formelle}
  \begin{block}{}
    \begin{itemize}
      \item Une démonstration rigoureuse et systématique
      \item Des axiomes et une suite d'étapes logiques
    \end{itemize}
  \end{block}

  \begin{exampleblock}{}
    \begin{itemize}
      \item Grande confiance dans la validité de la preuve
      \item Peut être vérifiée par un ordinateur
    \end{itemize}
  \end{exampleblock}

  \begin{alertblock}{}
    \begin{itemize}
      \item Processus souvent long et sujet à erreurs
    \end{itemize}
  \end{alertblock}
\end{frame}

\begin{frame}{Un assistant de preuve}
  \begin{block}{}
    \begin{itemize}
      \item Un logiciel pour manipuler les preuves formelles
      \item Coq, Isabelle, Agda, Lean, \dots
      \item Interaction Homme-Machine
    \end{itemize}
  \end{block}

  \begin{exampleblock}{}
    \begin{itemize}
      \item Facilite la création d'une preuve
      \item Permet de vérifier sa validité
    \end{itemize}
  \end{exampleblock}

  \begin{alertblock}{}
    \begin{itemize}
      \item Pas de construction automatique de preuve
    \end{itemize}
  \end{alertblock}
\end{frame}

\subsection{Automatisation et solveur SMT}
\begin{frame}{Automatisation}
  \begin{exampleblock}{}
    \begin{itemize}
      \item Gagner du temps
      \item Rendre les assistants de preuve plus accessibles
    \end{itemize}
  \end{exampleblock}

  \begin{block}<2>{Le cas du solveur SMT}
    \begin{itemize}
      \item Logiciel à part entière
      \item Permet de résoudre des problèmes de satisfiabilité
      \item Accepte des formules logiques
      \item Renvoie \texttt{SAT} ou \texttt{UNSAT}
    \end{itemize}
  \end{block}
\end{frame}

\begin{frame}{Un problème de communication}
  \begin{alertblock}{Des logiques différentes :}
    \begin{itemize}
      \item Assistant Coq : Calcul des constructions inductives
      \item Solveurs SMT : Logique du 1\ier{} ordre
    \end{itemize}
  \end{alertblock}

  \begin{exampleblock}{Du progrès, des tactiques de traduction :}
    \begin{thebibliography}{10}
      \beamertemplatearticlebibitems
      \bibitem{DBLP:conf/cpp/Blot0CPKMV23}
      L. D. de Prisque, C. Keller, V. Blot, \dots
      \newblock Compositional Pre-processing for Automated Reasoning in Dependent Type Theory
      \newblock CCP 2023.
    \end{thebibliography}
  \end{exampleblock}
\end{frame}

\subsection{Objectifs du stage}
\begin{frame}{Objectifs du stage}
  \begin{block}{Une traduction encore manuelle}
    \begin{itemize}
      \item Des tactiques pas toujours applicables
      \item Une solution : laisser l'utilisateur les appliquer
    \end{itemize}
  \end{block}

  \begin{exampleblock}{Vers l'automatisation de la traduction}
    \begin{itemize}
      \item Détecter les formules traduisibles
      \item Appliquer automatiquement les tactiques de traduction
    \end{itemize}
  \end{exampleblock}
\end{frame}

\section{Contribution}

% \subsection{Un langage de spécification}

% \begin{frame}{}
% \end{frame}

% \subsection{Un interpréteur}

% \begin{frame}{}
% \end{frame}

\section*{Conclusion}

\begin{frame}{Conclusion}
  % Keep the summary *very short*.
\end{frame}

\end{document}

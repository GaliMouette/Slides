\documentclass[french,usepdftitle=false,compress]{beamer}

\usepackage[french]{babel}
\usepackage[utf8]{inputenc}
\usepackage[T1]{fontenc}

\usepackage{minted}

\usepackage{tikz}
\usetikzlibrary{positioning,fit}

\mode<presentation>
{
  \usetheme{Dresden}
  \usecolortheme{crane}
  \setbeamertemplate{navigation symbols}{}
  \setbeamertemplate{footline}[frame number]
}

\title
{
  Application spontanée de transformations logiques en assistance au raisonnement automatique
}

\subtitle
{
  Stage de L3
}

\author[Alexis~CARRÉ]
{
  Alexis~CARRÉ \inst{1}\\[1em]
  {
    \tiny
    Sous la direction de :\\[-0.5em]
    Chantal~KELLER \inst{2} \hspace{1em} Louise~DUBOIS~DE~PRISQUE \inst{2}
  }
}

\institute[École Normale Supérieure de Lyon et Université Paris Saclay]
{
  \inst{1}
  École Normale Supérieure de Lyon
  \and
  \inst{2}
  Laboratoire Méthodes Formelles\\
  Université Paris Saclay
}

\date
{
  4 septembre 2023 : Soutenance
}

% Structuring a talk is a difficult task and the following structure
% may not be suitable. Here are some rules that apply for this
% solution:

% - Exactly two or three sections (other than the summary).
% - At *most* three subsections per section.
% - Talk about 30s to 2min per frame. So there should be between about
%   15 and 30 frames, all told.

% - A conference audience is likely to know very little of what you
%   are going to talk about. So *simplify*!
% - In a 20min talk, getting the main ideas across is hard
%   enough. Leave out details, even if it means being less precise than
%   you think necessary.
% - If you omit details that are vital to the proof/implementation,
%   just say so once. Everybody will be happy with that.

\begin{document}

\begin{frame}[noframenumbering, plain]
  \titlepage
\end{frame}

\section{Introduction}

\subsection{Les assistants de preuve}

\begin{frame}{La preuve formelle}
  \begin{block}{}
    \begin{itemize}
      \item Une démonstration rigoureuse et systématique
      \item Des axiomes et une suite d'étapes logiques
    \end{itemize}
  \end{block}

  \begin{exampleblock}{}
    \begin{itemize}
      \item Grande confiance dans la validité de la preuve
      \item Peut être vérifiée par un ordinateur
    \end{itemize}
  \end{exampleblock}

  \begin{alertblock}{}
    \begin{itemize}
      \item Processus souvent long et sujet à erreurs
    \end{itemize}
  \end{alertblock}
\end{frame}

\begin{frame}{Un assistant de preuve}
  \begin{block}{}
    \begin{itemize}
      \item Un logiciel pour manipuler les preuves formelles
      \item Coq, Isabelle, Agda, Lean, \dots
      \item Interaction Homme-Machine
    \end{itemize}
  \end{block}

  \begin{exampleblock}{}
    \begin{itemize}
      \item Facilite la création d'une preuve
      \item Permet de vérifier automatiquement la validité d'une preuve
    \end{itemize}
  \end{exampleblock}

  \begin{alertblock}{}
    \begin{itemize}
      \item Pas de construction automatique de preuve
    \end{itemize}
  \end{alertblock}
\end{frame}

\begin{frame}{Exemple : $x + 0 = x$}
  \inputminted{coq}{basic.v}
\end{frame}

\begin{frame}{Un peu d'architecture}
  \begin{center}
    \begin{tikzpicture}
      \node (A) {Assistant de preuve};

      \node[below left=0.5cm of A] (T) {Tactiques};
      \node[below=0cm of T,align=left,text width=4cm,font=\tiny] (TT) {
        \begin{itemize}
          \item Manipulent les preuves
          \item Appliquées par l'utilisateur
        \end{itemize}
      };
      \node[rounded corners, draw=black, fit=(T) (TT)] {};

      \node[below right=0.5cm of A] (N)  {Noyau};
      \node[below=0cm of N,align=left,text width=4cm,font=\tiny] (NN) {
        \begin{itemize}
          \item Vérifie la validité des preuves
          \item Appelé automatiquement
          \item \textbf{Ne fais pas de recherche}
          \item \textbf{Indépendant du reste}
        \end{itemize}
      };
      \node[rounded corners, draw=black, fit=(N) (NN)] {};

      \node[rounded corners,draw=black,fit=(A) (T) (TT) (N) (NN), inner sep=10pt] {};
    \end{tikzpicture}
  \end{center}

  \vfill

  \pause
  \begin{block}{}
    Architecture simplifiée par rapport à Coq, mais retrouvée dans notre prototype
  \end{block}
\end{frame}

\subsection{Automatisation et solveur SMT}
\begin{frame}{Automatisation}
  \begin{exampleblock}{}
    \begin{itemize}
      \item Gagner du temps
      \item Rendre les assistants de preuve plus accessibles
    \end{itemize}
  \end{exampleblock}
  \vfill
  \pause
  \inputminted{coq}{basic_lia.v}
\end{frame}

\begin{frame}{Toujours plus d'automatisation}
  \begin{block}{Le cas du solveur SMT}
    \begin{itemize}
      \item Logiciel à part entière
      \item Permet de résoudre des problèmes de décision
      \item Reçoit une formule logique et des contraintes
      \item Renvoie \texttt{SAT}\footnote{Il existe des valeurs telles que la formule est vraie}, \texttt{UNSAT}\footnote{La formule est toujours fausse} ou \texttt{UNKNOWN}
    \end{itemize}
  \end{block}

  \begin{block}{Exemple de formule :}
    \begin{math}
      sorted(t,i,j) = \forall i',j'. i \leq i' \land i' \leq j' \land j' \leq j \Rightarrow t[i'] \leq t[j']
    \end{math}
  \end{block}
\end{frame}

\begin{frame}{Encore un peu d'architecture}
  \begin{block}{Du point du vue du solveur SMT}
    \center
    \begin{tikzpicture}
      \node[rectangle] (forall) at (0,0) {$\forall i',j'.$};
      \node[rectangle] (a) [right=0 of forall] {$i \leq i'$};
      \node[rectangle] (and1) [right=0 of a] {$\land$};
      \node[rectangle] (b) [right=0 of and1] {$i' \leq j'$};
      \node[rectangle] (and2) [right=0 of b] {$\land$};
      \node[rectangle] (c) [right=0 of and2] {$j' \leq j$};
      \node[rectangle] (imply) [right=0 of c] {$\Rightarrow$};
      \node[rectangle] (d) [right=0 of imply] {$t[i'] \leq t[j']$};

      \onslide<2->{
        \node[rectangle,fill=blue!20] (forall) at (0,0) {$\forall i',j'.$};
        \node[rectangle,fill=blue!20,minimum width=1em,minimum height=1em,draw=black] (Blue) [below=0.5 of forall] {};
        \node (BlueText) [right=0 of Blue] {Instantiation};
      }
      \onslide<3->{
        \node[rectangle,fill=red!20] (and1) [right=0 of a] {$\land$};
        \node[rectangle,fill=red!20] (and2) [right=0 of b] {$\land$};
        \node[rectangle,fill=red!20] (imply) [right=0 of c] {$\Rightarrow$};
        \node[rectangle,fill=red!20,minimum width=1em,minimum height=1em,draw=black] (Red) [below=0.25 of Blue] {};
        \node (RedText) [right=0 of Red] {Logique};
      }
      \onslide<4->{
        \node[rectangle,fill=green!20] (a) [right=0 of forall] {$i \leq i'$};
        \node[rectangle,fill=green!20] (b) [right=0 of and1] {$i' \leq j'$};
        \node[rectangle,fill=green!20] (c) [right=0 of and2] {$j' \leq j$};
        \node[rectangle,fill=green!20] (d) [right=0 of imply] {$t[i'] \leq t[j']$};
        \node[rectangle,fill=green!20,minimum width=1em,minimum height=1em,draw=black] (Green) [below=0.25 of Red] {};
        \node (GreenText) [right=0 of Green] {Théories};
      }
    \end{tikzpicture}
  \end{block}
\end{frame}

\begin{frame}{Exemple : [1,2,3] est trié}
  \begin{block}{$\lnot sorted([1,2,3],0,2)$}
    \begin{itemize}
      \item \alt<1>
            {$\lnot \forall i',j'. i \leq i' \land i' \leq j' \land j' \leq j \Rightarrow t[i'] \leq t[j']$}
            { $\lnot \forall i',j'. 0 \leq i' \land i' \leq j' \land j' \leq 2 \Rightarrow [1,2,3][i'] \leq [1,2,3][j']$
            }
      \item<3-> \texttt{UNSAT}
    \end{itemize}
  \end{block}
\end{frame}

\begin{frame}{Un problème de communication}
  \begin{alertblock}{Des logiques différentes :}
    \begin{itemize}
      \item Assistant Coq : Calcul des constructions inductives
      \item Solveurs SMT : Logique du 1\ier{} ordre
    \end{itemize}
  \end{alertblock}

  \only<1>
  {
    \begin{columns}
      \begin{column}{0.3\textwidth}
        \begin{block}{\centering Immédiat :}
          \centering
          $\exists x, x + 0 = x$
        \end{block}
      \end{column}

      \begin{column}{0.3\textwidth}
        \begin{block}{\centering À traduire :}
          \centering
          $\exists R, \forall x, R(x,x)$
        \end{block}
      \end{column}
    \end{columns}
  }
  \only<2>
  {
    \begin{exampleblock}{Du progrès, des tactiques de traduction :}
      \begin{thebibliography}{10}
        \beamertemplatearticlebibitems
        \bibitem{DBLP:conf/cpp/Blot0CPKMV23}
        L. D. de Prisque, C. Keller, V. Blot, \dots
        \newblock Compositional Pre-processing for Automated Reasoning in Dependent Type Theory
        \newblock CPP 2023.
      \end{thebibliography}
    \end{exampleblock}
  }
\end{frame}

\subsection{Objectifs du stage}
\begin{frame}{Objectifs du stage}
  \begin{block}{Un ordre de traduction fixe}
    \begin{itemize}
      \item Résultat d'éxpérimentation
      \item Changer d'ordre en fonction du contexte ?
    \end{itemize}
  \end{block}

  \begin{exampleblock}{Vers un ordre automatiquement déterminé}
    \begin{itemize}
      \item Détecter les formules traduisibles
      \item Appliquer automatiquement les tactiques de traduction
    \end{itemize}
  \end{exampleblock}
\end{frame}

\section{Réalisation}


% \begin{frame}{Détecter ? (10)}
%   \begin{columns}
%     \begin{column}{0.5\textwidth}
%       \begin{block}{\centering En amont}
%         \begin{itemize}
%           \item Forme de la formule
%           \item Permet d'éviter des applications
%         \end{itemize}
%       \end{block}
%     \end{column}

%     \begin{column}{0.5\textwidth}
%       \begin{block}{\centering En aval}
%         \begin{itemize}
%           \item Résultat d'une tactique
%           \item Permet de combiner des tactiques
%         \end{itemize}
%       \end{block}
%     \end{column}
%   \end{columns}

%   \vfill

%   \begin{alertblock}{}
%     Insuffisant pour l'automatisation, besoin d'exprimer la position
%   \end{alertblock}
% \end{frame}

\subsection{Un langage de spécification}
\begin{frame}{Aperçu}
  \begin{block}{\mintinline{ocaml}|TEq (TGoal, TSomeHyp)|}
    Le but est-il égal à une hypothèse ?
  \end{block}

  \begin{block}{\mintinline{ocaml}|TIs (TSomeHyp, TAnd (TVar "A", TVar "B"))|}
    Existe-t-il une hypothèse de la forme $A \land B$ ?
  \end{block}
\end{frame}

\begin{frame}{Syntaxe}
  \inputminted[fontsize=\small]{ocaml}{syntax_var.ml}
  \vfill
  \inputminted[fontsize=\small]{ocaml}{syntax_form.ml}
  \vfill
  \inputminted[fontsize=\small]{ocaml}{syntax.ml}
\end{frame}

\begin{frame}{Sémantique et interprétation}
  \begin{columns}
    \begin{column}{0.5\textwidth}
      \begin{block}{Variables}
        \begin{itemize}
          \item \mintinline{ocaml}|TGoal|
          \item \mintinline{ocaml}|TSomeHyp|
        \end{itemize}
      \end{block}
    \end{column}

    \begin{column}{0.5\textwidth}
      \begin{block}{Formes}
        \begin{itemize}
          \item \mintinline{ocaml}|TVar|, \mintinline{ocaml}|TArr|, \mintinline{ocaml}|TAnd|, \mintinline{ocaml}|TOr|, \mintinline{ocaml}|TTop|, \mintinline{ocaml}|TBottom|
          \item \mintinline{ocaml}|TDiscard|
          \item \mintinline{ocaml}|TMetaVar|
        \end{itemize}
      \end{block}
    \end{column}
  \end{columns}

  \begin{block}{Autres}
    \begin{itemize}
      \item \mintinline{ocaml}|TEq|
      \item \mintinline{ocaml}|TIs|
      \item \mintinline{ocaml}|TContains|
    \end{itemize}
  \end{block}
\end{frame}

\section{Conclusion}

% \begin{frame}{Autres réalisations}
%   \begin{block}{Quelques tactiques}
%     \begin{itemize}
%       \item Élimination de la conjonction
%       \item Auto
%       \item Commute
%     \end{itemize}
%   \end{block}
% \end{frame}

\begin{frame}
  \begin{block}{Conclusion}
    \begin{itemize}
      \item Un langage de spécification
      \item Une implémentation dans un prototype
      \item Des perspectives
    \end{itemize}
  \end{block}

  \begin{block}{Perspectives}
    \begin{itemize}
      \item Travaux futurs de Louise
      \item Implémentation dans Coq avec LTac2
      \item Étudier l'ordre et son impact
    \end{itemize}
  \end{block}
\end{frame}

\end{document}

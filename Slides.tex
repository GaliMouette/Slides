\documentclass[french,usepdftitle=false,compress]{beamer}

\usepackage[french]{babel}
\usepackage[utf8]{inputenc}
\usepackage[T1]{fontenc}

\usepackage{minted}

\usepackage{tikz}
\usetikzlibrary{positioning}

\mode<presentation>
{
  \usetheme{Dresden}
  \usecolortheme{crane}
  \setbeamertemplate{navigation symbols}{}
  \setbeamertemplate{footline}[frame number]
}

\title
{
  Application spontanée de transformations logiques en assistance au raisonnement automatique
}

\subtitle
{
  Stage de L3
}

\author[Alexis~CARRÉ]
{
  Alexis~CARRÉ \inst{1}\\[1em]
  {
    \tiny
    Sous la direction de :\\[-0.5em]
    Chantal~KELLER \inst{2} \hspace{1em} Louise~DUBOIS~DE~PRISQUE \inst{2}
  }
}

\institute[École Normale Supérieure de Lyon et Université Paris Saclay]
{
  \inst{1}
  École Normale Supérieure de Lyon
  \and
  \inst{2}
  Laboratoire Méthodes Formelles\\
  Université Paris Saclay
}

\date
{
  4 septembre 2023 : Soutenance
}

% Structuring a talk is a difficult task and the following structure
% may not be suitable. Here are some rules that apply for this
% solution:

% - Exactly two or three sections (other than the summary).
% - At *most* three subsections per section.
% - Talk about 30s to 2min per frame. So there should be between about
%   15 and 30 frames, all told.

% - A conference audience is likely to know very little of what you
%   are going to talk about. So *simplify*!
% - In a 20min talk, getting the main ideas across is hard
%   enough. Leave out details, even if it means being less precise than
%   you think necessary.
% - If you omit details that are vital to the proof/implementation,
%   just say so once. Everybody will be happy with that.

\begin{document}

\begin{frame}[noframenumbering, plain]
  \titlepage
\end{frame}

\section{Introduction}

\subsection{Les assistants de preuve}

\begin{frame}{La preuve formelle}
  \begin{block}{}
    \begin{itemize}
      \item Une démonstration rigoureuse et systématique
      \item Des axiomes et une suite d'étapes logiques
    \end{itemize}
  \end{block}

  \begin{exampleblock}{}
    \begin{itemize}
      \item Grande confiance dans la validité de la preuve
      \item Peut être vérifiée par un ordinateur
    \end{itemize}
  \end{exampleblock}

  \begin{alertblock}{}
    \begin{itemize}
      \item Processus souvent long et sujet à erreurs
    \end{itemize}
  \end{alertblock}
\end{frame}

\begin{frame}{Un assistant de preuve}
  \begin{block}{}
    \begin{itemize}
      \item Un logiciel pour manipuler les preuves formelles
      \item Coq, Isabelle, Agda, Lean, \dots
      \item Interaction Homme-Machine
    \end{itemize}
  \end{block}

  \begin{exampleblock}{}
    \begin{itemize}
      \item Facilite la création d'une preuve
      \item Permet de vérifier sa validité
    \end{itemize}
  \end{exampleblock}

  \begin{alertblock}{}
    \begin{itemize}
      \item Pas de construction automatique de preuve
    \end{itemize}
  \end{alertblock}
\end{frame}

\begin{frame}{Exemple : $x + 0 = x$}
  \inputminted{coq}{basic.v}
\end{frame}

\begin{frame}{Un peu d'architecture}
  \begin{block}{Les tactiques :}
    \begin{itemize}
      \item Manipulent les preuves
      \item Appliquées par l'utilisateur
    \end{itemize}
  \end{block}

  \begin{block}{Le noyau :}
    \begin{itemize}
      \item Vérifie la validité des preuves
      \item Ne fait pas de raisonnement
      \item \textbf{Inépendant du reste}
    \end{itemize}
  \end{block}
\end{frame}

\subsection{Automatisation et solveur SMT}
\begin{frame}{Automatisation}
  \begin{exampleblock}{}
    \begin{itemize}
      \item Gagner du temps
      \item Rendre les assistants de preuve plus accessibles
    \end{itemize}
  \end{exampleblock}
  \vfill
  \pause
  \inputminted{coq}{basic_lia.v}
\end{frame}

\begin{frame}{Toujours plus d'automatisation}
  \begin{block}{Le cas du solveur SMT}
    \begin{itemize}
      \item Logiciel à part entière
      \item Permet de résoudre des problèmes de décision
      \item Reçois une formule logique et des contraintes
      \item Renvoie \texttt{SAT} ou \texttt{UNSAT}
    \end{itemize}
  \end{block}

  \begin{block}{Exemple de formule :}
    \begin{math}
      sorted(t,i,j) = \forall i',j'. i \leq i' \land i' \leq j' \land j' \leq j \Rightarrow t[i'] \leq t[j']
    \end{math}
  \end{block}
\end{frame}

\begin{frame}{Encore un peu d'architecture}
  \begin{block}{Du point du vue du solveur SMT}
    \center
    \begin{tikzpicture}
      \node[rectangle] (forall) at (0,0) {$\forall i',j'.$};
      \node[rectangle] (a) [right=0 of forall] {$i \leq i'$};
      \node[rectangle] (and1) [right=0 of a] {$\land$};
      \node[rectangle] (b) [right=0 of and1] {$i' \leq j'$};
      \node[rectangle] (and2) [right=0 of b] {$\land$};
      \node[rectangle] (c) [right=0 of and2] {$j' \leq j$};
      \node[rectangle] (imply) [right=0 of c] {$\Rightarrow$};
      \node[rectangle] (d) [right=0 of imply] {$t[i'] \leq t[j']$};

      \onslide<2->{
        \node[rectangle,fill=blue!20] (forall) at (0,0) {$\forall i',j'.$};
        \node[rectangle,fill=blue!20,minimum width=1em,minimum height=1em,draw=black] (Blue) [below=0.5 of forall] {};
        \node (BlueText) [right=0 of Blue] {Instantiation};
      }
      \onslide<3->{
        \node[rectangle,fill=red!20] (and1) [right=0 of a] {$\land$};
        \node[rectangle,fill=red!20] (and2) [right=0 of b] {$\land$};
        \node[rectangle,fill=red!20] (imply) [right=0 of c] {$\Rightarrow$};
        \node[rectangle,fill=red!20,minimum width=1em,minimum height=1em,draw=black] (Red) [below=0.25 of Blue] {};
        \node (RedText) [right=0 of Red] {Logique};
      }
      \onslide<4->{
        \node[rectangle,fill=green!20] (a) [right=0 of forall] {$i \leq i'$};
        \node[rectangle,fill=green!20] (b) [right=0 of and1] {$i' \leq j'$};
        \node[rectangle,fill=green!20] (c) [right=0 of and2] {$j' \leq j$};
        \node[rectangle,fill=green!20] (d) [right=0 of imply] {$t[i'] \leq t[j']$};
        \node[rectangle,fill=green!20,minimum width=1em,minimum height=1em,draw=black] (Green) [below=0.25 of Red] {};
        \node (GreenText) [right=0 of Green] {Théorie};
      }
    \end{tikzpicture}
  \end{block}
\end{frame}


\begin{frame}{Un problème de communication}
  \begin{alertblock}{Des logiques différentes :}
    \begin{itemize}
      \item Assistant Coq : Calcul des constructions inductives
      \item Solveurs SMT : Logique du 1\ier{} ordre
    \end{itemize}
  \end{alertblock}

  \begin{exampleblock}{Du progrès, des tactiques de traduction :}
    \begin{thebibliography}{10}
      \beamertemplatearticlebibitems
      \bibitem{DBLP:conf/cpp/Blot0CPKMV23}
      L. D. de Prisque, C. Keller, V. Blot, \dots
      \newblock Compositional Pre-processing for Automated Reasoning in Dependent Type Theory
      \newblock CCP 2023.
    \end{thebibliography}
  \end{exampleblock}
\end{frame}

\subsection{Objectifs du stage}
\begin{frame}{Objectifs du stage}
  \begin{block}{Une traduction encore manuelle}
    \begin{itemize}
      \item Des tactiques pas toujours applicables
      \item Une solution : laisser l'utilisateur les appliquer
    \end{itemize}
  \end{block}

  \begin{exampleblock}{Vers l'automatisation de la traduction}
    \begin{itemize}
      \item Détecter les formules traduisibles
      \item Appliquer automatiquement les tactiques de traduction
    \end{itemize}
  \end{exampleblock}
\end{frame}

\section{Contribution}

% \subsection{Un langage de spécification}

% \begin{frame}{}
% \end{frame}

% \subsection{Un interpréteur}

% \begin{frame}{}
% \end{frame}

% \Autres contributions

\section*{Conclusion}

\begin{frame}{Conclusion}
  % Keep the summary *very short*.
\end{frame}

\end{document}
